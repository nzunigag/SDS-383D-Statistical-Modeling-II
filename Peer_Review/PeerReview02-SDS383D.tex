\documentclass[11 pt]{article} 
\usepackage[left=2cm, top=2cm, right=2cm, bottom=2.5cm, footskip=.5cm]{geometry}
\usepackage{url}
\usepackage{enumitem}

\author{Natalia Zuniga-Garcia}
\title{Peer Review 2}
\date{April 6, 2018}

\begin{document}

\maketitle

\section{Description}

I went trough the documentation available in the student's GitHub page by April 6, 2018.

\begin{itemize}
	\item Student evaluated:  Sareh Kouchaki (\url{https://github.com/Sarehkch/Statistical-Modeling-II})
	\item Assignment reviewed: Sections 2 and 3

\end{itemize}
	


\section{General Comments}
\begin{itemize}
	\item You posted your solutions in a PDF format and attached all the code you used.
	\item You presented the solution of the exercises providing a wide description of the procedure, which makes easier to understand the process you followed.
	\item Your solutions are very straightforward and easy to follow.
	\item I really liked that you added a link or the references you used for the solutions. This can help other readers if they follow your results.
	\item You explained broadly all the math derivations including the steps.
	\item You GitHub page is very organized, I would encourage you to add more description of the documentation in a README.md file in case someone else, out of the course, would like to check on your results and on your code. 

	
\end{itemize}

\section{Specific Comments}
\begin{itemize}
		\item \textbf{Code Style Evaluation:} I know this is not a coding course but I checked on your code and I would like to provide some feedback on the coding style. I based my comments on \url{http://adv-r.had.co.nz/Style.html}.
		\begin{itemize}
			\item When assigning values to objects, try to use consistently $<-$ instead of =. I noted that in Section 2 you used =, generally is better to use $<-$, in Section 3 you used it correctly.
			\item Try to place spaces around all infix operators (=, +, -, $<-$, etc.). 
			\item You separated sections of your code using $\#\#\#\#\#\#\#\#\#\#\#\#\#\#\#$, this is very positive because heps to separate exercise and main code part, which makes it easier to follow.
			\item Another positive comment is that you explained every part of the code, commenting on the respective section.
		\end{itemize}
		\item \textbf{Exercise 2.2:} Optionally, you can add the interpretation of the parameters at the end of your derivation, this will help you to see how the multivarite and univariate form change.
		\item \textbf{Exercise 3.7:} I had issues understanding this question and you provided a wide solution that helped me go trough it. 

\end{itemize}

\section{Summary}
Dear Sareh, 

I tried to provide you good feedback. I tried check in detail your documentation so you can improve, and this helped me as well. I learned from your code and derivation. I hope you learn from my review.


\end{document}